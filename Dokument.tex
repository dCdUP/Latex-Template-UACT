%------------------ vorlage.tex ------------------------------------------------
%
%
%-------------------------------------------------------------------------------


%------------------ Präambel ---------------------------------------------------
\documentclass[envcountsame, envcountchap, deutsch]{i-studis}
\usepackage{parskip} 
\usepackage[utf8]{inputenc}
\usepackage[xindy,style=indexgroup]{glossaries}

\usepackage[a4paper]{geometry}
\usepackage[english, ngerman]{babel}
\usepackage{float}
\usepackage{nameref}

\usepackage[pdftex]{graphicx}
\usepackage{epstopdf}
\usepackage{subcaption}

\usepackage{listings}
\usepackage{makecell}

\usepackage[german, ruled, vlined]{algorithm2e}
\usepackage{amssymb, amsfonts, amstext, amsmath}
\usepackage{array}
%\usepackage[skip=10pt]{caption}
\usepackage[usenames, dvipsnames]{color}
\usepackage{xcolor}
%\usepackage{sectsty}
\usepackage{textcomp}
\usepackage{booktabs}
\usepackage{wrapfig}
\usepackage{titlesec}
\usepackage{blindtext}
\usepackage[shortcuts]{extdash}
% \usepackage{enumerate}
\usepackage{enumitem}
\usepackage{svg}

% Zitierpakete
\usepackage[style=apa,backend=biber, maxcitenames=2]{biblatex} %quotation style
% \DeclareLanguageMapping{german}{english}
\usepackage[pdftex, plainpages=false, breaklinks=true]{hyperref}
\usepackage[babel, german=quotes]{csquotes}
\addbibresource{Literatur.bib}

\usepackage{makeidx}
\usepackage{multicol}
\setlength{\tabcolsep}{0.5em}                  % for horizontal padding in tables

% Einstellungen für Code-Listings
\lstset{
        basicstyle=\ttfamily\scriptsize,       % print whole listing small and in monospace
        keywordstyle=\color{blue}\bfseries,    % underlined bold black keywords
        identifierstyle=,                      % nothing happens
        commentstyle=\color{red},              % white comments
        stringstyle=\ttfamily,                 % typewriter type for strings
        showstringspaces=false,                % no special string spaces
        framexleftmargin=7mm, 
        tabsize=3,
        showtabs=false,
        frame=single, 
        rulesepcolor=\color{blue},
        numbers=left,
        linewidth=146mm,
        xleftmargin=8mm,
        captionpos=b
}

\graphicspath{ {./Abbildungen/} }

\hypersetup{
    colorlinks,
    linkcolor={black},
    citecolor={blue!50!black},
    urlcolor={blue!80!black}
}

\makeindex

\pagestyle{myheadings}
\setlength{\textheight}{1.1\textheight}

\makenoidxglossaries

\newglossaryentry{Unique_Name}{
    name={Unique Name},
    description={description of glossary entry}
}



%------------------ Titelseite -------------------------------------------------
\begin{document}
\setcounter{tocdepth}{4}
\setcounter{secnumdepth}{4}
\title{} % Titel der Wissenschaftlichen Arbeit
\project{Seminararbeit}
% \degree{Bachelor of Science (B.Sc.)}                    % Nur bei Abschlussarbeit!
\address{im Studiengang Wirtschaftsinformatik an der\\Hochschule Trier}

\author{Niklas Metzen}
\supervisorFirst{Prof.\ Dr.\ Maximilian Mustermann}
\matrikelnummer{}
\fachbereich{Wirtschaft}
\studiengang{Wirtschaftsinformatik}
\submitdate{\the\day.\the\month.\the\year}

\mytitlepage{}

%------------------ Vorwort, Kurzfassung, Verzeichnisse ------------------------
\frontmatter
\tableofcontents					% Inhaltsverzeichnis
\listoffigures                        % Abbildungsverzeichnis
\listoftables                           % Tabellenverzeichnis
%------------------ Kapitel ----------------------------------------------------
\mainmatter{}

%------------------ Kapitel 1 -------------------------------------------------------------

\chapter{Chapter} %name of chapter
\section{Section} %name of section
%\cite{} % citation unique name
\subsection{Subsection} %name of subsection
Unstructured list
\begin{itemize} %unstructured list
    \item %Object and description with
\end{itemize}

%------------------ Kapitel n -------------------------------------------------------------

\vspace{1ex}
Figure
\begin{figure} %start of figure
    \includegraphics[width=\textwidth]{comparison_variance_two_points.drawio.png} %source for picture
    \caption{Eine erhöhte Varianz ist von Vorteil}\label{fig:var_max} % Caption of figure
\end{figure}

Reference of fiqure~\ref{fig:var_max}
Reference of source~\cite{6065061}

Center aligned text
\begin{align} %center aligned text
    \textnormal{Varianz Ratio}_{i} = \frac{var_{i}}{\sum _{i=1}^{n} {var_{i}}} % align 
\end{align}

Center aligned Table
\begin{table} %pretty table
    \centering
    \begin{tabular}{lcccccl}
        \toprule 
        & \multicolumn{1}{c}{$k=3$} & \multicolumn{1}{c}{$k=7$}
        \\\cmidrule(lr){2-2}\cmidrule(lr){3-3}
        $Klasse$ & $Anzahl$ & $Anzahl$ \\
        $A$    &   $2$     & $5$       \\
        $B$ &   $1$     & $3$       \\\bottomrule
    \end{tabular}
    \caption{Auflistung der Klassen der $k$ nächsten Nachbarn für den unklassifizierten Datenpunkt in Abbildung}\label{tab:majority_rule}
\end{table}

Ordered list
\begin{enumerate} %ordered list
    \item One
    \item Two
\end{enumerate}

Glossary
$\gls{Unique_Name}$

%------------------ Literaturverzeichnis & Index -------------------------------

\backmatter{}
\printbibliography{}
\printindex
\printnoidxglossaries										% Index (optional)
%------------------ Anhänge ----------------------------------------------------

\end{document}